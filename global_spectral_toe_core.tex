\documentclass[11pt]{article}

\usepackage{amsmath,amssymb,amsfonts}
\usepackage{physics}
\usepackage{geometry}
\usepackage{hyperref}
\geometry{margin=1in}

\title{A Spectral--Relational Theory of Everything\\
\large Core Mathematical Construction}
\author{Grzegorz Olbryk}
\date{}

\begin{document}
\maketitle

\begin{abstract}
I present the core mathematical construction of a spectral--relational
Theory of Everything (TOE), in which physical structure emerges from
global spectral properties of a relational system rather than from
local dynamics on a predefined spacetime.

This document provides the axioms, definitions, and derived structures
of the theory at the fundamental level. It does not attempt to provide
a complete effective description of all observed physical phenomena.
In particular, strongly chaotic many--body dynamics are not assumed
to arise directly at the fundamental level.

Validation tests, scope limitations, and falsification results,
including Random Matrix Theory (RMT) analysis, are presented separately
in companion documents and in the public repository.
\end{abstract}

\section{Foundational Axioms}

\textbf{Axiom A1 (Spectral Ontology).} \\
Physical reality is fully encoded in a finite set of positive spectra
\begin{equation}
\{\lambda^{(a)}_n\}_{n \in \mathbb{N}}, \qquad \lambda^{(a)}_n > 0,
\end{equation}
where the index \(a\) labels relational sectors.

\textbf{Axiom A2 (Relationality).} \\
Only relations between sectors are physically meaningful.
Absolute scales are unobservable.

\textbf{Axiom A3 (Log--Spectra).} \\
The physically relevant quantities are logarithmic spectra
\begin{equation}
\ell^{(a)}_n := \log \lambda^{(a)}_n .
\end{equation}

\textbf{Axiom A4 (Spectral Gaps).} \\
All dynamics are encoded in spectral gaps
\begin{equation}
\Delta^{(a)}_{mn} := \ell^{(a)}_m - \ell^{(a)}_n .
\end{equation}

These axioms define a global, background--independent structure.
No spacetime, metric, Hamiltonian, or action principle is assumed.

\section{Global Spectral Functional}

I define a global spectral functional
\begin{equation}
\mathcal{L} = \sum_{a,b} w_{ab} \sum_{m,n}
\left(\ell^{(a)}_m - \ell^{(b)}_n\right)^2 ,
\end{equation}
where \(w_{ab}\) are relational weights.

The logarithmic form ensures scale invariance and compatibility
with Axiom A2. The specific choice of weights is constrained
by stability and covariance considerations but is not claimed
to be mathematically unique.

\section{Vacuum Configuration}

A vacuum is defined as a stationary point of \(\mathcal{L}\)
with respect to variations of \(\ell^{(a)}_n\).

The vacuum condition yields a finite set of fixed points
\begin{equation}
c_a^\ast = \langle \ell^{(a)} \rangle ,
\end{equation}
defined up to a global additive constant.

These fixed points characterize relational sectors of the theory.
They do not by themselves constitute a complete effective
description of particle interactions.

\section{Emergent Mass Scale}

A dimensionless mass scale arises from dominant spectral gaps:
\begin{equation}
m \sim \left| c_a^\ast - c_b^\ast \right| .
\end{equation}

This definition yields a hierarchy of mass scales without
introducing fundamental mass parameters.
Mass is intrinsically nonlocal and global in this framework.

\section{Emergent Time and Irreversibility}

Iterative evolution of spectral states,
\begin{equation}
\phi_{n+1} = T \phi_n ,
\end{equation}
where \(T\) is a global transport operator compatible with
the spectral structure, generates irreversible behavior
whenever the dominant eigenvalue satisfies \(|\lambda_0| < 1\).

This produces an intrinsic arrow of time without assuming
microscopic reversibility or local causality.

\section{Entropy}

A spectral entropy can be defined as
\begin{equation}
S = - \sum_i p_i \log p_i ,
\end{equation}
where \(p_i\) are normalized spectral weights.

Monotonic behavior of \(S\) under forward spectral evolution
provides a statistical characterization of time consistent
with the emergent arrow.

\section{Gauge--Like Sector Structure}

Relational sectors characterized by distinct \(c_a^\ast\)
exhibit stable clustering behavior.

Certain minimal configurations are compatible with structures
resembling
\[
U(1) \times SU(2) \times SU(3),
\]
but this identification is structural and suggestive rather
than a proven necessity.

The present construction does not exclude the existence of
other mathematically consistent sector decompositions.

\section{On Physical Constants}

Numerical values of the fixed points \(c_a^\ast\) may exhibit
proximity to known dimensionless physical parameters.

Such numerical proximity is noted as suggestive but is not
claimed as a definitive identification with experimentally
measured constants at this stage.

A complete mapping to observable couplings requires an
explicit effective--theory construction.

\section{Gravity and Long--Range Structure}

The global, relational nature of mass implies an attractive
long--range interaction at large scales.

While this framework suggests an emergent gravitational
behavior, a full derivation of Einstein--type dynamics is
beyond the scope of the present work.

\section{No--Go Results}

The theory admits several structural no--go results:

\begin{itemize}
\item No local mass: restricting to local subregions eliminates
spectral gaps.
\item No local time: local spectra remain unit--modulus.
\item No fundamental ergodicity: global relational constraints
limit chaotic mixing.
\end{itemize}

These results follow directly from the axioms and do not depend
on specific model realizations.

\section{Relation to Validation Tests}

Random Matrix Theory analyses of concrete excitation operators
constructed within this framework show pseudo--integrable,
mixed spectral statistics rather than universal GOE behavior.

This result constrains the interpretation of the theory but
does not contradict its role as a fundamental framework.

Strong quantum chaos, if present in nature, is expected to
emerge only at the level of effective many--body excitations
built upon the present construction.

\section{Conclusion}

This document presents the core mathematical structure of a
spectral--relational Theory of Everything.

The theory is internally consistent, background--independent,
and falsifiable. Its limits are explicit: it does not claim
to directly reproduce all effective physical phenomena.

Further work is required to construct effective theories of
excitations and to establish detailed connections with
phenomenology.

\appendix
\section{Example Numerical Fixed Points}

For a specific choice of relational weights and truncation,
the vacuum conditions yield numerical fixed points of the form:
\begin{align}
c_1^\ast &\approx 0.2358, \\
c_2^\ast &\approx 0.3142, \\
c_3^\ast &\approx 0.4501.
\end{align}

These values are presented solely as explicit examples of solutions
of the vacuum equations for a particular model realization.

Different choices of relational weights or truncations
generically lead to different numerical fixed points.

No claim is made that these numerical values are uniquely
determined by the axioms alone or that they correspond
one--to--one with experimentally measured constants.

Their role is illustrative and structural rather than
phenomenological.

\end{document}
